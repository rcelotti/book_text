% !TeX program = xelatex
%\documentclass[a4paper,italian]{article}
%\documentclass[preview,convert,italian]{standalone}
%\documentclass[convert,italian]{standalone}
\documentclass[
border={0.5cm 0.5cm 0.5cm 0.5cm}, %left right bottom top
%border={0.1cm 0.1cm 0.1cm 0.1cm}, %left right bottom top
preview,
italian]{standalone}
\usepackage[italian]{babel}
\usepackage[T1]{fontenc}
\usepackage[utf8]{inputenc}
% \usepackage{erewhon}
\usepackage{lettrine}
\usepackage{GoudyIn}
\usepackage[x11names]{xcolor} 
\usepackage{psvectorian}
\usepackage{pgfornament}
\usepackage[paperwidth=8cm]{geometry}
% \usepackage[paperwidth=13cm]{geometry}

\renewcommand{\LettrineFontHook}{\color{black}\GoudyInfamily{}}
% \LettrineTextFont{\itshape}
% \setcounter{DefaultLines}{3}%

\newcommand{\ornamentbreak}{%
    \begin{center}
        \psvectorian[width=5cm]{88}%
    \end{center}%
}

\begin{document}
    
    \pagenumbering{gobble}
    
%    \begin{flushright}
%        \today
%    \end{flushright}
    
%    \lettrine[lhang=0, findent=0.2em, nindent=-0.0em, lines=3]{C}{iao} Bubizzi...\\
%    Oggi ho mangiato la pasta senza pista ma con un sacco di pepperoni!\\
%    Gigibagigi.

    % Use:
    %   @CL@ number of lines lettrine will use
    %   @C@ for capital (first) letter used by lettrine
    %   @CT@ for lettrine first word
    %   @TXT@ for text
    % these will be replaced by main script with input text
    
    \lettrine[lhang=0, findent=0.2em, nindent=-0.0em, lines=@CL@]{@C@}{@CT@} @TXT@
    
    \ornamentbreak
    
\end{document}
